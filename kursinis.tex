\documentclass{VUMIFPSkursinis}
\usepackage{algorithmicx}
\usepackage{algorithm}
\usepackage{algpseudocode}
\usepackage{amsfonts}
\usepackage{amsmath}
\usepackage{bm}
\usepackage{caption}
\usepackage{color}
\usepackage{float}
\usepackage{graphicx}
\usepackage{listings}
\usepackage{subfig}
\usepackage{wrapfig}
\usepackage{fontspec}
\usepackage{todonotes}
\newcommand{\TD}[1]{\todo[inline, color=red!40]{#1}}

% Titulinio aprašas
\university{Vilniaus universitetas}
\faculty{Matematikos ir informatikos fakultetas}
\department{Programų sistemų bakalauro studijų programa}
\papertype{Kursinis darbas}
\title{Raft Algoritmo Ra Praplėtimų Tyrimas}
\titleineng{Investigation of Raft Algorithm Ra Extensions}
\status{3 kurso 1 grupės studentas}
\author{Edvardas Dlugauskas}
% \secondauthor{Vardonis Pavardonis}   % Pridėti antrą autorių
\supervisor{dr. Karolis Petrauskas}
\date{Vilnius – \the\year}

% Nustatymai
\setmainfont
[Path = ./palem/,
 Extension = .ttf,
 UprightFont = *-nm,
 BoldFont = *-bd,
 ItalicFont = *-it,
 BoldItalicFont = *-bi]
{Palem}
\bibliography{bibliografija}

\begin{document}
\maketitle
\listoftodos
\tableofcontents

\sectionnonum{Įvadas}
%Įvade apibūdinamas darbo tikslas, temos aktualumas ir siekiami rezultatai.
%Darbo įvadas neturi būti dėstymo santrauka. Įvado apimtis 1–2 puslapiai.

Paskirstyta sistema yra skaičiavimų elementų (toliau - mazgų) rinkinys, kuris yra naudojamas kaip vientisa sistema~\cite{steen_distributed_2017}. Paskirstytose sistemose konsensuso algoritmai yra reikalingi būsenų sinchronizavimui tarp sistemos mazgų. Konsensuso algoritmas turi užtikrinti, kad sistema būtų prieinama vieno ar daugiau mazgų gedimo ar neprieinamumo atvejais~\cite{ongaro_consensus}. 

Šiame darbe išnagrinėti Raft~\cite{ongaro_consensus} konsensuso algoritmo praplėtimai naudojami Ra~\cite{rabbitmqra} bibliotekoje. \textbf{Darbo tikslas yra patikrinti, ar Raft algoritmas, praplėstas Ra naudojamais praplėtimais vis dar atitinka pagrindines Raft algoritmo savybes}: lyderio rinkimų saugumą; lyderio žurnale esamų įrašų uždarumą modifikacijoms; visų žurnalų įrašų atitikimą; lyderio žurnalo pilnumą; ir sistemos kaip būsenų mašinos saugumą~\cite{ongaro_consensus}. \TD{ateity pridėti nuorodą į detalesnius savybių aprašymus, į dėstymo skyrių}

\textbf{Šio darbo uždaviniai yra:}
\begin{itemize}
    \item Išnagrinėti Raft algoritmą ir biblioteką Ra.
    \item Pateikti esamos literatūros apie Raft algoritmą ir Ra įgyvendinimo apžvalgą.
    \item Išanalizuoti Ra padarytus praplėtimus ir jų įtaką Raft algoritmui.
    \item Pateikti formalią specifikaciją Ra algoritmo praplėtimams.
\end{itemize}

\TD{trumpai apie darbo rezultatą}

%Medžiagos darbo tema dėstymo skyriuose pateikiamos nagrinėjamos temos detalės:
%pradinė medžiaga, jos analizės ir apdorojimo metodai, sprendimų įgyvendinimas,
%gautų rezultatų apibendrinimas. Šios dalies turinys labai priklauso nuo darbo
%temos. Skyriai gali turėti poskyrius ir smulkesnes sudėtines dalis, kaip
%punktus ir papunkčius.
%
%Medžiaga turi būti dėstoma aiškiai, pateikiant argumentus. Tekstas dėstomas
%trečiuoju asmeniu, t.y. \enquote{rašoma} ne „aš manau“, bet „autorius mano“, „autoriaus
%nuomone“. Reikėtų vengti informacijos nesuteikiančių frazių, pvz., „...kaip jau
%buvo minėta...“, „...kaip visiems žinoma...“ ir pan., vengti grožinės literatūros
%ar publicistinio stiliaus, gausių metaforų ar panašių meninės išraiškos
%priemonių.

\section{Raft algoritmo apžvalga}

Raft yra grupė algoritmų naudojamų konsensusui pasiekti paskirstytoje sistemoje. Raft korektiškumas yra įrodytas formalia TLA$^+$ specifikacija~\cite{ongaro_consensus} ir naudojimu industrijoje -- Raft naudojamas etcd\footnote{https://etcd.io/}, Consul\footnote{https://www.consul.io/docs/internals/consensus.html} ir CockroachDB\footnote{https://www.cockroachlabs.com/docs/stable/architecture/replication-layer.html} projektuose. 

Raft algoritmo kontekste sistema yra būsenų mašina, kuri priima iš klientų komandų įrašus. Kiekvienas įrašas yra komandos, norimos įvykdyti sistemai, aprašymas. Įvykdant komandą, sistema pakeičia savo vidinę būseną ir grąžina klientui rezultatą -- savo atnaujintą būseną ar jos dalį. Atitinkamai, paskirstyta sistema susideda iš mazgų, kurie irgi yra būsenų mašinos, turinčios savo įrašų žurnalus ir vidines būsenas. Toks paskirstytos sistemos išdėstymas leidžia pasiekti aukštą gedimų toleravimo ir greičio lygį~\cite{ongaro_consensus, steen_distributed_2017}.

Paskirstytos sistemos veikimo metu gali būti situacijų, kai kelių mazgų būsenos yra skirtingos. Pavyzdžiui, kai vienas iš mazgų trumpam laikui tampa neprieinamas. Mazgų būsenos gali išsiskirti, ir iš skirtingų mazgų klientui gali būti grąžinami skirtingi rezultatai. Tai pažeistų sistemos kaip vienos didelės būsenų mašinos vientisumą. Todėl būtina sinchronizuoti būsenas tarp mazgų ir tuo pačiu metu leisti sistemai veikti toliau jei mažesnioji dalis mazgų neveikia ar yra nepasiekiama. Tam reikia išspręsti konsensuso problemą -- kaip mazgai gali pasiekti susitarimą kokia yra jų bendra būseną, net kai kai kurie iš mazgų gali būti neveikiantys ar jų duomenys pasenę?

Raft algoritmas buvo sukurtas kaip lengviau suprantama alternatyva tuo metu labiausiai paplitusiam Paksos\footnote{http://lamport.azurewebsites.net/pubs/paxos-simple.pdf} konsensuso algoritmui~\cite{ongaro_consensus, diego_designing_2016}. Raft autoriai, atlikus vartotojų tyrimą, parodė, kad Raft algoritmas yra lengviau suprantamas negu Paksos, iš ko seka mažesni įgyvendinimo kaštai, mažesnis galimų klaidų skaičius ir paprastesnis bazinio algoritmo praplėtimas~\cite{ongaro_consensus}. Raft leidžia užtikrinti sistemos gyvumą ir saugumą kol veikia daugiau negu pusė paskirstytos sistemos mazgų. Raft algoritmas remiasi lyderio išrinkimu -- vienas iš mazgų tampa lyderiu ir yra laikoma kad lyderio žurnalas yra tiesos šaltinis. Visos klientų užklausos yra siunčiamos lyderiui, kuris persiunčia įrašus kitiems mazgams (sekėjams). Kai dauguma mazgų turi informaciją apie naują įrašą, mazgai užfiksuoja įrašą savo žurnale ir įvykdo įraše nurodytą komandą. Jei nustoja veikti sekėjas, sistema tęsia darbą, kol yra veikiančių mazgų dauguma. Nustojus veikti lyderiui, nesulaukę lyderio gyvumo patvirtinimo, kiekvienas mazgas po atsitiktinio dydžio laiko pradeda rinkimus. Išsirinkus naują lyderį sistemą gali tęsti darbą~\cite{ongaro_consensus}. 

Raft specifikacija garantuoja kad Raft algoritmas tenkina sekančias savybes~\cite{ongaro_consensus}: 

\begin{enumerate}
\item Lyderio išrinkimas: duotam lyderio terminui galiausiai bus išrinktas vienas lyderis.
\item Lyderio žurnalo uždarumas modifikacijoms: lyderis niekada nekeičia ir netrina įrašų savo žurnale, jis gali tik pridėti naujų įrašų.
\item Žurnalų įrašų atitikimas: jeigu dviejų mazgų žurnalai turi įrašą su tuo pačiu indeksu ir terminu, tai tie įrašai yra lygūs.
\item Lyderio žurnalo pilnumas: jeigu žurnalo įrašas buvo užfiksuotas, tai visų sekančių terminų lyderiai turės tą užfiksuotą įrašą savo žurnale.
\item Sistemos kaip būsenos mašinos saugumas: jei mazgas įvykdė žurnalo įrašą, joks kitas mazgas niekada neįvykdys skirtingo įrašo su tuo pačiu indeksu.
\end{enumerate}

Toliau detaliau apžvelgsime kaip Raft užtikrina tai, kad galiausiai būtų išrinktas mazgų lyderis ir kaip lyderis perduoda įrašus kitiems mazgams.

\subsection{Raft lyderio išrinkimas}

Pagal Raft, mazgas kuris yra lyderis periodiškai siunčia savo sekėjams tuščias įrašo pridėjimo užklausas (\texttt{AppendEntries}), taip užtikrinant savo ,,širdies plakimą'' (angl. ,,heartbeat'')~\cite{ongaro_consensus}. Jeigu sekėjas kažkurį laiką negauna užklausų iš lyderio, sekėjas pradeda naujus rinkimus ir pasisiūlo kaip kandidatas į naują lyderį. Laikas, po kurio sekėjas pradės naujo termino rinkimus vadinamas \textit{rinkimų laiku} (angl. \textit{election timeout}). Rinkimų laikas yra atsitiktinė reikšmė iš pasirinkto laiko intervalo, kas padeda greičiau išrinkti naują lyderį kai yra daugiau negu vienas kandidatas~\cite{ongaro_consensus}.

Kad pradėtų rinkimus, sekėjas tampa kandidatu, balsuoja šiame rinkimų termine už save, ir išsiunčia visiems kitiems mazgams prašymo balsuoti užklausas (\texttt{RequestVote}). Mazgas per rinkimų terminą gali balsuoti tik už vieną mazgą. Galiausiai: arba kandidatas laimi rinkimus, arba kitas mazgas laimi rinkimus, arba lyderis nėra išrenkamas ir pradedamas naujas balsavimas~\cite{ongaro_consensus}.

\subsection{Raft įrašų perdavimas}

Išrinktas lyderis aptarnauja klientus. Iš kliento lyderis gauną užklausą, kurioje nusakytas veiksmas, kurį reikia atlikti paskirstytai sistemai kaip būsenų mašinai. Gautą žinutę lyderis prideda į savo žurnalą kaip naują įrašą, tada asinchroniškai išsiunčia įrašo pridėjimo užklausas sekėjams. Kai dauguma sekėjų sėkmingai prideda įrašą į savo žurnalą, lyderis įvykdo atitinkamą veiksmą savo būsenų mašinai ir grąžina rezultatą klientui~\cite{ongaro_consensus}. 

Net jei dalis mazgų nustojo veikti ar yra lėti, lyderis pakartotinai siunčia įrašų pridėjimo užklausas tiems sekėjams, kurie dar nepatvirtino įrašo gavimą. Galiausiai visi mazgai turės visus žurnalo įrašus. Šis pakartotinio informacijos siuntimo procesas gali vykti net jau lyderiui išsiuntus atsakymą klientui~\cite{ongaro_consensus}.

\section{Raft algoritmo įgyvendinimas Ra}

Šioje dalyje yra apžvelgiama Raft algoritmo įgyvendinimas Ra. Ra yra atviro kodo Raft algoritmo biblioteka, naudojama atviro kodo žinučių brokerio RabbitMQ\footnote{https://www.rabbitmq.com/}. Ra biblioteka yra naudojama RabbitMQ komandos paskirstytoms eilėms įgyvendinti, tačiau yra laisvai prieinama ir gali būti naudojama kaip bendra Raft biblioteka. Ra yra įgyvendinta Erlang\footnote{https://www.erlang.org/} kalba.

Nors didžioji dalis Raft algoritmo savybių yra įgyvendintos pagal Raft specifikaciją, dėl Erlang kalbos ypatybių ir konkrečių naudojimo scenarijų Ra išsiskiria nuo specifikacijos, pavyzdžiui, žurnalo perdavimo ir gedimų aptikimo algoritmais~\cite{rabbitmqra}.

Pirmoje dalyje apžvelgsime Raft praplėtimus, padarytus Ra.

\subsection{Ra įgyvendinti Raft algoritmo praplėtimai}

Dėl Erlang kalbos ypatybių, konkretaus naudojimo scenarijaus ir siekio optimizuoti Raft algoritmo veikimo laiką ir naudojamų duomenų kiekį~\cite{rabbitmqra}, Ra išsiskiria nuo specifikacijos kelių savybių įgyvendinimu.

\subsubsection{Ra žurnalo perdavimo algoritmas}

Ra nenaudoja žurnalo įrašų perdavimo užklausų mazgų gyvybingumui užtikrinti kai visų mazgų žurnalai yra sinchronizuoti. Kai vienas iš mazgų yra pasenęs, tai yra, to mazgo paskutinis sinchronizuotas įrašas yra ankstesnis negu paskutinis įrašas, siųstas lyderio tam mazgui, \texttt{AppendEntries} užklausa yra daroma, bet rečiau, negu pasiūlyta Raft specifikacijoje. Jei skirtumas tarp paskutinio sinchronizuoto įrašo ir paskutinio siųsto įrašo yra labai didelis, mazgo apkrovai sumažinti užklausos nėra siunčiamos~\cite{rabbitmqra}. 

Žurnalų įrašų perdavimas yra vykdomas asinchroniškai, atsakymai yra grupuojami~\cite{rabbitmqra}.

Ra \texttt{AppendEntries} užklausa naudoja tris nestandartinius laukus: \texttt{last\_index}, \texttt{last\_term}, \texttt{next\_index}. \texttt{last\_index} ir \texttt{last\_term} naudojami lyderio naujam \texttt{commit\_index} apskaičiuoti. \texttt{next\_index} naudojamas kai sekėjui nepavyksta atnaujinti žurnalo, tada šio lauko reikšmė nurodo nuo kurio žurnalo įrašo reikia tęsti žurnalo įrašų siuntimą~\cite{rabbitmqra}.

\subsubsection{Ra gedimų aptikimo algoritmas}

Ra nenaudoja Raft aprašyto gedimo aptikimo algoritmo. Ra naudoja gimtąjį Erlang mazgų stebėjimo infrastruktūrą stebėti ar mazgai nenustojo veikti ir Aten\footnote{https://github.com/rabbitmq/aten} biblioteką mazgų susisiekimo klaidoms aptikti~\cite{rabbitmqra}.

Erlang monitoriai leidžia greitai aptikti mazgų neveikimą kodo klaidos metimo atveju (angl. ,,crash''). Stebėtojui yra išsiunčiama \texttt{'DOWN'} žinutė~\cite{ericsson_erlang_processes_2016}. Erlang monitoriai užtikrina greitą lyderio mazgo neveikimo aptikimą jo sekėjais ir taip iš dalies pakeičia Raft širdies plakimo mechanizmą~\cite{rabbitmqra}. Aten biblioteka leidžia aptikti tinklo paskirstymo problemas, kaip mazgų nepasiekiamumą per tinklą. Kai Aten įtaria kad vienas iš mazgų yra nepasiekiamas, yra išsiunčiama \texttt{nodedown} žinutė mazgo stebėtojui. Atitinkamai, jei mazgas ,,atsigauna'' ir atrodo vėl veikiantis, yra išsiunčiama \texttt{nodeup} žinutė~\cite{rabbitmq_aten_2020}. Kartu Erlang monitoriai ir Aten biblioteka, anot Ra autorių, užtikrina mazgų gyvumą ir pakeičia Raft aprašytą širdies plakimo mechanizmą~\cite{rabbitmqra}.

Ra bibliotekoje visi mazgai kurie naudoja Raft algoritmą yra įgyvendinti kaip būsenų mašinos Erlang modulio \texttt{gen\_statem}\footnote{http://erlang.org/doc/man/gen\_statem.html} pagalba. 
Pirmiausia būsenų mašina  prenumeruoja gauti pranešimus apie mazgų statusų pokyčius su \texttt{net\_kernel:monitor\_nodes/1}\footnote{http://erlang.org/doc/man/net\_kernel.html\#monitor\_nodes-1}. Kai vienas iš mazgų tampa neprieinamas, mazgas gauna \texttt{'DOWN'} pranešimą ir įvykdo atitinkamą Raft algoritmo logiką.

\section{Raft Ra praplėtimų korektiškumas}
\TD{parodyti kad (dalis) praplėtimų yra korektiški, galimai TLA+ pagalba}

\subsection{Praplėtimas - Log replication}

\subsection{Praplėtimas - Failure Detection}

\sectionnonum{Rezultatai ir išvados}

%Rezultatų ir išvadų dalyje turi būti aiškiai išdėstomi pagrindiniai darbo
%rezultatai (kažkas išanalizuota, kažkas sukurta, kažkas įdiegta) ir pateikiamos
%išvados (daromi nagrinėtų problemų sprendimo metodų palyginimai, teikiamos
%rekomendacijos, akcentuojamos naujovės).

\printbibliography[heading=bibintoc]  % Šaltinių sąraše nurodoma panaudota
% literatūra, kitokie šaltiniai. Abėcėlės tvarka išdėstomi darbe panaudotų
% (cituotų, perfrazuotų ar bent paminėtų) mokslo leidinių, kitokių publikacijų
% bibliografiniai aprašai.  Šaltinių sąrašas spausdinamas iš naujo puslapio.
% Aprašai pateikiami netransliteruoti. Šaltinių sąraše negali būti tokių
% šaltinių, kurie nebuvo paminėti tekste.

% \sectionnonum{Sąvokų apibrėžimai}
\sectionnonum{Santrumpos}
%Sąvokų apibrėžimai ir santrumpų sąrašas sudaromas tada, kai darbo tekste
%vartojami specialūs paaiškinimo reikalaujantys terminai ir rečiau sutinkamos
%santrumpos.

\appendix  % Priedai

% Prieduose gali būti pateikiama pagalbinė, ypač darbo autoriaus savarankiškai
% parengta, medžiaga. Savarankiški priedai gali būti pateikiami ir
% kompaktiniame diske. Priedai taip pat numeruojami ir vadinami. Darbo tekstas
% su priedais susiejamas nuorodomis.

\end{document}
